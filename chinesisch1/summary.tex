\documentclass[UTF8]{ctexart}
\usepackage[UTF8]{ctex}

\begin{document}

\section{Ausprache}

\subsection{Ausnahmen}

\begin{itemize}
    \item Folgt nach dem Wort 不 bù eine Silbe in einem vierten Ton, dann wird bù nicht bù,
    sondern bú im zweiten Ton ausgesprochen. Und den Tonwechsel muss man
    schreiben. Z.B. bú shì
    \item wenn zwei Silben im dritten Ton nebeneinander stehen die erste Silbe im zweiten
    Ton gelesen ABER NICHT geschrieben wird
    \item 还是 kann auch hái shi gelesen werden.
\end{itemize}

\subsection{Schwacher Ton}

Siehe S.77 in Zhongguohua

\section{Grammatik}

\subsection{Satzstruktur}

Immer Subjekt, Angabe, Prädikate, Objekt, Modalpartikel.
吗 ist beispielsweise ein Modalpartikel. 怎么 kommt immer vor dem Verb. Ein Objekt kann auch ein Objektsatz sein, welcher wiederrum nach derselben Struktur aufgebaut ist.

\subsection{Mehr}

Siehe Dokumente auf OLAT.

\end{document}
