\documentclass[UTF8]{article}
\usepackage[]{url}

\begin{document}

\section{EBNF}

\begin{itemize}
    \item wenn es keine Klammern gibt: Aufreihung bindet stärker als Auswahl
\end{itemize}

\section{Protected in Java}

\begin{itemize}
    \item Protected: Subclasses oder aktuelles Package können zugreifen
    \item Bei Access von anderem Package: kann nur auf aktuellen Typ zugreifen (siehe protected_1.png, protected_2.png)
\end{itemize}

\section{Operator Precedence}

\begin{itemize}
    \item Check \url{http://web.deu.edu.tr/doc/oreily/java/langref/ch04_14.htm}
\end{itemize}

\section{More}

\begin{itemize}
    \item Multi-line string with """..."""
    \item a = 0; a = a++; then a = 0 => a = 0; tmp = a; a = a + 1; a = tmp;
    \item with booleans: \& does not short-circuit, \&\& does short-circuit
    \item https://stackoverflow.com/a/25182306/13445631
\end{itemize}

\section{Imports}

Importieren einer einzelnen Klasse gibt der Klasse einenhohen Stellenwert:

\begin{itemize}
    \item Mit import foo.* gilt: eine Klasse mit dem selben Name in dieser Package verdeckt die importierte Klasse
    \item Mit import foo.className gilt: die Klasse mit dem selben Namen verdeckt nicht die importierte Klasse
\end{itemize}

\end{document}
