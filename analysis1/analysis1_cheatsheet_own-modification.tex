% Basic stuff
\documentclass[a4paper,10pt]{article}
\usepackage[utf8]{inputenc}
\usepackage[german]{babel}

% 3 column landscape layout with fewer margins
\usepackage[landscape, left=0.75cm, top=1cm, right=0.75cm, bottom=1.5cm, footskip=15pt]{geometry}
\usepackage{flowfram}
\ffvadjustfalse
\setlength{\columnsep}{1cm}
\Ncolumn{3}

% define nice looking boxes
\usepackage{tcolorbox}

% a base set, that is then customised
\tcbset {
  base/.style={
    boxrule=0mm,
    leftrule=1mm,
    left=1.75mm,
    arc=0mm, 
    fonttitle=\bfseries, 
    colbacktitle=black!10!white, 
    coltitle=black, 
    toptitle=0.75mm, 
    bottomtitle=0.25mm,
    title={#1}
  }
}

\definecolor{brandblue}{rgb}{0.6, 1, 0.8}
\newtcolorbox{mainbox}[1]{
  colframe=brandblue, 
  base={#1}
}

\newtcolorbox{subbox}[1]{
  colframe=black!20!white,
  base={#1}
}

% Mathematical typesetting & symbols
\usepackage{amsthm, mathtools, amssymb} 
\usepackage{marvosym, wasysym}

% Tables
\usepackage{tabularx, multirow}
\usepackage{booktabs}
\renewcommand*{\arraystretch}{2}

% Make enumerations more compact
\usepackage{enumitem}
\setitemize{itemsep=0.5pt}
\setenumerate{itemsep=0.75pt}

% To include sketches
\usepackage{graphicx}


% Math helper stuff
\def\limn{\lim\limits_{n\to \infty}}
\def\sumk{\sum\limits_{k=1}^{\infty}}
\def\sumn{\sum\limits_{n=0}^{\infty}}
\def\R{\mathbb{R}}
\def\N{\mathbb{N}}
\def\dx{\text{ d}x}

\begin{document}


\subsection{Beweise}
\begin{subbox}{}
  Seien $f,g : [a,b] \to \R$ wobei f stetig, g beschränkt integrierbar mit $g(x) \geq 0 \forall x\in[a,b]$. Dann gibt es $\xi \in [a,b]$ mit:\\
  $\int_a^b f(x)g(x)dx = f(\xi)\int_a^b g(x)dx$\\
  Beweis, weil $g(x) \geq 0$:\\
  $\underset{y\in[a,b]}{\inf}g(y)f(x) \leq f(x)g(x) \leq \underset{x\in[a,b]}{\inf} f(x)g(x) \forall x \in [a,b]$\\
  $\underset{y\in[a,b]}{\inf}g(y)\int_a^b f(x)dx \leq \int_a^b g(x)f(x)dx \leq \underset{x\in[a,b]}{\inf} \int_a^b f(x)dx \, \forall x \in [a,b]$
\end{subbox}

\begin{subbox}{}
  Cauchy-Produkt von $e^x - 1 = x + x^2 + ...$ mit $\sumn \frac{c_n x^n}{n!}$ ist gleich x genau dann wenn $c_0=1$ und für alle $k\geq 2$ $\sum\limits_{i=0}^{k}\binom{k}{i} = 0$\\
  \begin{align*}
    CP &= \sum\limits_{i=0}^{\infty}a_i \sum\limits_{j=0}^{\infty}\frac{c_j x^j}{j!} = \sumn(\sum\limits_{j=0}^{n}a_{n-j}b_j)\\
    &= \sumn(\frac{c_0}{n!}x^n + \frac{c_1}{1!(n-1)!}x^n\\
    &+ \frac{c_2}{2!(n-2)!}x^n + ... + \frac{c_{n-1}}{(n-1)!1!} + 0)\\
    &= 0 + c_0x + \sum\limits_{n=2}^{\infty}(\sum\limits_{j=0}^{n-1}\frac{c_j}{j!(n-j)!}x^n)
  \end{align*}
  aber: 
  $\sum\limits_{j=0}^{n-1}\frac{c_j}{j!(n-j)!}x^n = \frac{1}{n!}\sum\limits_{j=0}^{n-1}\frac{n!}{j!(n-j)!}c_j = \frac{1}{n!}\sum\limits_{j=0}^{n-1}\binom{n}{j}c_j$\\
  Also $CP = c_0x + \sum\limits_{n=2}^{\infty}(\sum\limits_{j=0}^{n-1}\binom{n}{j}c_j)\frac{x^n}{n}$\\
  Was zeigt, dass $CP = x \Leftrightarrow c_0 = 1$ und $\sum\limits_{j=0}^{n-1}\binom{n}{j}c_j = 0, \, \forall n\geq 2$
\end{subbox}

\begin{subbox}{}
  Falls $\sumn a_n x^n$ positive Konvergenzradius besitzt, so ist der Konvergenzradius von $\sumn \frac{a_n x^n}{n!}$ gleich $+\infty$.\\
  Es gilt, dass $\limn \sup \sqrt[n]{|a_n|}$ existiert, also $\limn \sup \sqrt[n]{|a_n|} = \limn \underset{k\geq n}{\sup} \sqrt[k]{|a_k|}$\\
  Per definition ist der Konvergenzradius $\rho$ von $\sumn \frac{a_n}{n!}x^n$ (wo $\frac{a_n}{n!}=b_n$) gleich $+\infty$ gdw. $\limn \sup \sqrt[n]{|b_n|}=0$\\
  Wir zeigen also $\limn\sup \sqrt[n]{|b_n|} = \limn \underset{k\geq n}{\sup} \sqrt[k]{|b_k|} = \limn \underset{k\geq n}{\sup}|a_k|^\frac{1}{k}\cdot k!^{-\frac{1}{k}}$\\
  Mit der Stirling'schen Formel gilt:\\
  $k!^{-\frac{1}{k}} \approx (\frac{\sqrt{2\pi k}k^k}{e^k}) = \frac{(2\pi k)^{-\frac{k}{2}}e}{k} = \frac{e}{(\sqrt{2\pi k})^k\cdot k} \rightarrow 0$\\
  Ausserdem existiert $\limn\underset{k\geq n}{\sup}\sqrt[k]{|a_k|}$, wo $\underset{k\geq n}{\sup}\sqrt[k]{|a_k|} = c_n$ per Annahme und somit ist die Folge $c_n$ beschränkt. Ex existiert also eine $C\in\R$ mit $c_n \leq C$ für all n und damit\\
  \begin{align*}
    \limn\sup\sqrt[n]{|b_n|} &= \limn \underset{k\geq n}{\sup}|a_k|^\frac{1}{k}\cdot k!^{-\frac{1}{k}}\\
    &\leq \limn C\cdot k!^{-\frac{1}{k}}\\
    &C \limn \underset{k \geq n}{\sup}k!^{-\frac{1}{k}}\\
    &C \limn \sup n!^{-\frac{1}{n}}\\
    &C \limn n!^{-\frac{1}{n}}\\
    &=0
  \end{align*}

  Im vorletzten Schritt: falls der Grenzwert existiert ist lim inf gleich lim sup 
\end{subbox}


\begin{subbox}{}
  Zeige, dass f integrierbar ist:
  \begin{equation*}
    f(x) =
    \begin{cases*}
      0 & $x=0$ oder $x\in\R \backslash \mathbb{Q}$\\
      \frac{1}{q} & $x=\frac{p}{q}$ mit p,q$\in \mathbb{N}_{>0}$ teilerfremd.
    \end{cases*}
  \end{equation*}
  
  Wir wenden den Satz von Du-Bois an:
  Bemerke: jedes Intervall $[x_{i-1}, x_i]$ beinhaltet eine irrationale Zahl, solange $x_{i-1} \neq x_i$. Damis ist für jede Partition P von [0,1] die Untersumme s(f,P) gleich 0.\\
  Sei $\epsilon > 0$ und $n\in \N$ so dass $\frac{1}{n} < \frac{\epsilon}{2}$ und $B_n$ die Menge aller "komprimen Brüche" mit Nenner kleiner gleich n:\\
  $B_n = \{1, \frac{1}{2}, \frac{1}{3}, \frac{2}{3}, \frac{1}{4},\frac{3}{4}, \frac{1}{5}, ..., \frac{1}{n}, ..., \frac{n-1}{n}\}$, sei $m = |B_n|$\\
  wähle $k>m$ und $P = \{x_0, ..., x_k\}$ eine Partition mit: $\underset{1\leq i\leq k}{max}|x_i - x_{i-1} < \frac{\epsilon}{4m}$\\
  So eine Partition existiert immer wenn wir k gross genuig wählen. Mit der Notation aus der Vorlesung gilt somit $ P = P_{\frac{e}{4m}}$.\\
  Da $B_n$ aus m-vielen Elementen besteht, gibt es höchstens 2m-viele Intervalle $[x_{i-1}, x_i]$ die $B_n$ schneiden.\\
  Falls $x\notin B_n$, dann ist x entweder irrational (f(x) = 0), 0, oder ein Bruch $\frac{p}{q}$ mit p,q teilerfremd und q>n. Auf jeden Fall gilt dann $f(x) < \frac{1}{n}$.\\
  Wir schätzen die Obersumme S(f,P) von oben ab:
  \begin{align*}
    S(f,P) &= \sum\limits_{i=1}^{k}(x_i - x_{i-1})\cdot \underset{x\in[x_i, x_{i-1}]}{\sup}f(x)\\
    &= \underset{i: [x_{i-1}], x_i \cap B_n \neq \emptyset}{\sum}(x_i - x_{i-1})\cdot f_i \\
    &+ \underset{j: [x_{j-1}], x_j \cap B_n \neq \emptyset}{\sum}(x_j - x_{j-1})\cdot f_j
  \end{align*}
  Die erste Summe läuft höchstens über 2m Indizes, da es höchstens soviele Intervalle gibt die $B_n$ schneiden und es gilt $f_i \leq 1$ (weil $f(x) \leq 1$). In der Zweiten Summe können wir $f_i$ mit $\frac{1}{n}$ abschätzen, da das Intervall $B_n$ nicht schneidet und wir f(y) für Element ausserhalb von $B_n$ mit $\frac{1}{n}$ oben abgeschätzt haben. 
\end{subbox}
\begin{subbox}{}
  Ausserdem gilt:\\
  $\underset{i: [x_{i-1}, x_i]\cap B_n \neq \emptyset}{\sum}(x_i - x_{i-1}) \leq \sum_{j = 1}^{k}(x_j - x_{j-1}) = x_k - x_0 = 1 - 0 = 1$\\
  Wir werfen alle zusammen und erhalten:\\
  $S(f,P) \leq 2m\cdot \frac{\epsilon}{4m}\cdot 1+1\cdot\frac{1}{n} < \frac{\epsilon}{2}+\frac{\epsilon}{2} = \epsilon$\\
  Damit wäre $S(f,P) - s(f,P) < \epsilon, \, P = P_\frac{\epsilon}{4m}$ gezeigt und der Satz besagt, dass f integrierbar ist.
\end{subbox}

\begin{subbox}{tan(x)}
  \begin{enumerate}
    \item Zeige $tan: (-\frac{\pi}{2}, \frac{\pi}{2})$ ist strikt monoton wachsend:\\
    $tan'(x) = \frac{1}{\cos(x)^2} > 0$ und damit gezeigt, dass tan(x) strikt monoton wachsend ist.
    \item Zeige $\underset{x\to(\frac{\pi}{2})^-}{\lim}\tan(x) = +\infty$ und $\underset{x\to(\frac{\pi}{2})^+}{\lim}\tan(x) = -\infty$\\
    $\frac{\sin(x)}{\cos(x)}$ wo $\sin(\frac{\pi}{2}) = 1$ und $\cos{\frac{\pi}{2}} = 0$ aber $\cos(x) > 0$ für $(-\frac{\pi}{2}, 0)$ und $<0$ umgekehrt. 
    \item Zeige dass $\tan(x)$ bijektiv ist:\\
    Wir wissen, dass $\tan(x)$ injektiv ist, weil sie strikt monoton wachsend ist und stetig weil $\frac{\sin(x)}{\cos(x)}$ stetig sind. Zusammen mit den zwei Limes wissen wir aus dem Zwischenwertsatz, dass sie auch surjektiv ist. Damit ist sie bijektiv.
    \item Berechne die Ableitung von $\arctan(x)$:\\
    Sei $y = tan(x)$ es gilt dann $\arctan'(y) = \frac{1}{\tan'(x)} = \cos(x)^2$\\
    Bemerke: $y^2 = tan(x)^2 = \frac{1}{\cos(x)^2}-1$ und $\cos(x)^2 = \frac{1}{y^2 +1}$ und somit $\arctan'(x) = \frac{1}{y^2 + 1}$
  \end{enumerate}
\end{subbox}

\begin{subbox}{Beweis Young'sche Ungleichung}
  Seien $p,q > 1$ mit $\frac{1}{p}+\frac{1}{q} = 1$ gegeben. Zeige, dass für $a,b\geq 0$ die Ungleichung $ab\leq\frac{a^p}{p} + \frac{b^q}{q}$ gilt.\\
  Definiere $f:[0, \infty]\to \R$ durch $f(x):= \frac{1}{p}a^p + \frac{1}{q}x^q - ax$\\
  Die Ableitung ist gegeben durch $f'(x)= x^{q-1}-a$. Die Gleichung $f'(x) = 0$ hat die eindeutige Lösung $x_0 := a^\frac{1}{q-1}$\\
  $f(x_0) = \frac{1}{p}a^p + \frac{1}{q}a^\frac{q}{q-1} - aa^\frac{1}{q-1} = \frac{1}{p}a^p + \frac{1}{q}a^\frac{q}{q-1} - a^\frac{q}{q-1} = \frac{1}{p}a^p + \frac{1}{q}a^p - a^p = 0$\\
  Dabei haben wir $\frac{1}{p} + \frac{1}{q} = 1$ benutzt.\\
  Da $q>1$ gilt, ist $x^{q-1}$ eine streng monoton wachsende Funktion und es gilt $f'(x) < 0 $ für $x\in[0, x_0]$ sowie $f'(x) > 0$ für $x\in(x_0, \infty)$. Insbesondere ist f auf dem Intervall $[0,x_0)$ streng monoton fallend, auf dem Intervall $(x_0, \infty)$ streng monoton wachsend und die Funktion f nimmt an der Stelle $x_0$ ihr globales Minimum an. Für eine beliebige positive reelle Zahl $b\geq 0$ folgt:\\
  $\frac{1}{p}a^p + \frac{1}{q}b^q -ab = f(b) \geq f(x_0) = 0$\\
  und dies ist genau die Young'sche Ungleichung
\end{subbox}

\begin{subbox}{Riemann Integral}
  Zeige, dass $f:[0,1] \to \R$ integrierbar ist:
  \begin{equation*}
    f(x) =
    \begin{cases*}
      0 & $x=0$ oder $x\in\R \backslash \mathbb{Q}$\\
      \frac{1}{q} & $x=\frac{p}{q}$ mit p,q$\in \mathbb{N}_{>0}$ teilerfremd.
    \end{cases*}
  \end{equation*}

  Wenn u eine Treppenfunktion mit $u \leq f$ ist, dann gilt $\int_{a}^{b}u\,dx\leq 0$. Insbesondere $u =0\leq f$, $\int_{a}^{b}u \, dx = 0$. Die obere Schranke ist erreicht und dann erhalten wir $sup s(f) = 0$. Wir möchten $inf S(g) = 0$ beweisen. Dies genügt uns zu beweisen, dass f Riemann-integrierbar ist.\\
  
  Sei $\epsilon > 0$. Es gibt $n \in \N$ mit $\frac{1}{n} < \epsilon$. Wenn $x\in [0,1]$ mit $f(x)\geq \epsilon$ ist, dann gibt es höchstens $\sum_{k =1}^{n-1} k\leq \frac{n^2}{2}$ Werte x. Wir nennen diese Werte \textbf{schlechte Punkte}, und sonst \textbf{gute Punkte}. Sei $0 =  x_0 < x_1 < ... < x_N = 1$ eine Zerlegung wobei $N > \frac{n^2}{\epsilon}$. Diese Zerlegung ist gegeben durch $x_k \frac{k}{N}$. Zudem gilt $x_{k+1} - x_k = \frac{1}{N}$. Wir setzen: 
  \begin{equation*}
    v_N(x) = 
    \begin{cases*}
      1 & x slechter Punkt\\
      \epsilon & x guter Punkt
    \end{cases*}
  \end{equation*}

  Somit ist $\int_{a}^{b}v_N\, dx \leq n^2 \frac{1}{N} + N\epsilon\frac{1}{N} = \frac{n^2}{N} + e \leq 2\epsilon$. Ausserdem haben wir $v_n \geq f$. Wir erhalten:\\
  $0 = sup s(f) \leq inf S(f) \leq 2\epsilon$.\\
  Weil $\epsilon > 0$ beliebig ist, schliessen wir dass f Riemann-integrierbar ist und $\int_{0}^{1} f \, dx = 0$
\end{subbox}

\end{document}
