\documentclass[UTF8]{ctexart}
\usepackage[UTF8]{ctex}

\begin{document}

\section{Ausprache}

\subsection{Ausnahmen}

\begin{itemize}
    \item Folgt nach dem Wort 不 bù eine Silbe in einem vierten Ton, dann wird bù nicht bù,
    sondern bú im zweiten Ton ausgesprochen. Und den Tonwechsel muss man
    schreiben. Z.B. bú shì
    \item wenn zwei Silben im dritten Ton nebeneinander stehen die erste Silbe im zweiten
    Ton gelesen ABER NICHT geschrieben wird
    \item 还是 kann auch hái shi gelesen werden.
\end{itemize}


\subsection{不}

Obige Ausnahme gilt für 不. Ausserdem kann 不 auch im neutralen Ton gelesen werden (bspw. bei 对不起).

\subsection{一}

一 is pronounced in the first tone when it stands alone. It is pronounced in the fourth tone when it precedes a first, second, or third tone. However, it is pronounced in the second tone when it precedes a fourth tone.

When 一 (yī) appears as an ordinal number (as in "first"), or as a number in a series, address, or date, it is pronounced without the tone change (regular first tone "yī")

\begin{itemize}
    \item 一 (yī)
    \item 一百 (yìbǎi)
    \item 一万 (yíwàn)
\end{itemize}

Wenn auf 一 ein ZEW folgt, wird es im vierten Ton ausgesprochen.

Wenn man eine Telefonnummer sagt, darf man nicht yī sagen, sondern muss yāo sagen (und schreiben 幺).

\subsection{两}

两 wird generell verwendet als eine Zähleinheit. 二百 wird nur verwendet falls exakt 200 gesagt werden soll. 两百 ist auch möglich und wird generell bevorzugt. 二百零一 ist nicht möglich, nur 两百零一 ist möglich.

\subsection{Zahlen}

Nullen hintendran darf man weglassen. Ansonsten immer mit 零 "auffüllen". Bei Zehnern darf man Zahl am Ende weglassen.

\begin{itemize}
    \item 101: 一百零一
    \item 120: 一百二(十)
\end{itemize}

Siehe auch Video auf OLAT (2022-10-17).

Man kann auch 一万万 anstelle von 一亿 verwenden.


\subsection{Schwacher Ton}

Siehe S.77 in Zhongguohua für richtige Aussprache und Höhe.

\section{Grammatik}

\subsection{Farben}

Basisfarben können ohne 色 geschrieben werden, andere Farben immer mit 色. Bspw. 米色 aber nicht 米. Auch bei zweisilbigen Farben braucht es kein 的.

\subsection{的 und 得}

的 kann bei einsilbigen Attributen weggelassen werden (z.B. Farben).

得 ist ein Modalkomplement, beschreibt ein Verb. Immer Verb + 得 + Komplement. Da 得 immer nach einem Verb kommen muss, muss man das Verb für das Objekt wiederholen. Bspw. 他(说)汉语说得很快。

Es gibt einen Unterschied zwischen 他说很快的汉语 (schnelles Chinesisch) und 他说汉语说得很快 (sprich schnell Chinesisch).

Das erste Verb kann normalerweise weggelassen werden (他汉语说得很好).

\subsection{Satzstruktur}

Immer Subjekt, Angabe, Prädikate, Objekt, Modalpartikel.
吗 ist beispielsweise ein Modalpartikel. 怎么 kommt immer vor dem Verb. Ein Objekt kann auch ein Objektsatz sein, welcher wiederrum nach derselben Struktur aufgebaut ist.

\subsection{Mehr}

Siehe Dokumente auf OLAT.

\subsection{给}

给 kann verwendet werden um ein Objekt in die Angabe zu schieben. Beispielsweise gibt es in 你给我发短信吧 zwei Objekte, deshalb können wir ein Objekt (我) mit 给 in die Angabe stellen. Die Angabe ist dann 给我. Siehe auch Video auf OLAT (2022-10-17).

\subsection{Verben mit zwei Objekten}

Die Verben 给,问 und 教 können zwei Objekte haben.

Beispiel: 老师教学生汉字。\\
他问老师问题。

\subsection{Richtungen}

Dasselbe für Norden, Süden, etc.:

\begin{itemize}
    \item Im Norden von: 北方
    \item nördlich von: 北边
\end{itemize}

Für Nordosten, Südosten, etc. schreibt man zuerst W/E und erst dann N/S. 方 kann weggelassen werden wenn schon 2 Silben vorhanden sind (bspw. 西藏在中国西南).

\subsection{Prozente}

人口百分之八十是中国人。 80 Prozent der Bevölkerung sind Chinesen.\\
Mit Komma: 百分之二十二点五。 22,5 Prozent.

\subsection{Adjektivprädikate}

Siehe auch Video. Adjektive brauchen 很,太。。。了,非常. Ansonsten ist bspw. 你的中文好 komparativ. 你的中文很好 ist dann nicht komparativ. Um herauszufinden ob wirklich 很 gemeint ist auf Aussprache achten (wenn wirklich "sehr" gemeint ist, wird 很 stärker betont).

\subsection{Nominalprädikate}

Bspw. 他16岁. 岁 hat eine prädikative Funktion.

Bei der Verneinung braucht man ein 是. 他不是16岁. 他不16岁 ist falsch.

Auch beim Geld (bspw. 一条蓝裤子一百二十块钱, kein 是)

\subsection{很}

多 und 少 immer mit 很.

\subsection{圣祠}

todo

\subsection{ZEW}

ZEW verwenden bei:

\begin{itemize}
    \item Zahlwort 三,十,五十
    \item Demonstrativpronomen 这, 那
    \item Fragewort (welche/r/s), 哪
    \item Fragewort (wie viel) 几 (BEI 多少 DARF KEIN ZEW VERWENDET WERDEN)
\end{itemize}

Liste:

\begin{itemize}
    \item 个: allgemein (bspw. 哪个学生)
    \item 本: Bücher, alle Dinge die wie Bücher gebunden sind (几本书)
    \item 种: Sprachen (三种语言)
    \item 张: flache (dünne) Gegenstände (四张照片)
    \item 位: höflich für Personen (一位老师)
    \item 家: Firmen (三家公司)
    \item 件: Kleidung / Oberteil (毛衣,衬衫,上衣,T-恤)
    \item 位: Personen, respektvoller (三位老师)
    \item 本: Bücher, alle Dinge die wie Bücher gebunden sind (几本书)
    \item 口:Personen im gleichen Haushalt
    \item 双:Paare (Schuhe)
    \item 条:Dinge die lang sind (Schlange, Hose, Rock)
    \item 顶:Hüte
    \item 套:Set (西服)
    \item 块:Stücke (Kuchen, 手表)
    \item 副:Brillen, Handschuhe
\end{itemize}

ZEW mit zwei ist immer 两+ZEW. Nie 二.

\subsection{Kleidung}

Anziehen: 穿,戴,系.

\begin{itemize}
    \item 穿:Generell.
    \item 戴:Accessories (Hüte, Brillen, etc.)
    \item 系:tragen (umbinden)
\end{itemize}

\subsection{认识}

认识 wird mit Personen verwendet, 知道 mit Dingen. Es gibt aber Ausnahmen:

\begin{itemize}
    \item 我认识XY汉字.
\end{itemize}

\subsection{Alter}

Für Kinder: 几岁
Jugendliche und Erwachsene: 多大
älteren Menschen: 多大岁数

\section{Geographie}

Seite 70-72 im Buch Provinzen mit Hauptstädten und Aussprache. Karte auf Seite 48.

\section{Geld}

Bei zwei Einheiten ist die zweite Einheit optional. Bspw. 一块五(毛)

Wenn es nur eine Einheit ist, muss man noch 钱 hinzufügen. Bspw. 五百块钱

\end{document}
